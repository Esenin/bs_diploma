\documentclass{spisok-article}

\title{Рандомизированный алгоритм при обработке данных ультразвуковых исследований
}

\author{Сенин И. И.,
  студент кафедры системного программирования СПбГУ,
  i.senin@2012.spbu.ru
}

\begin{document}

\maketitle

\begin{abstract}

Ультразвуковая томография нашла широкое применение в медицинской практике. Для выского качества изображения требуется большое количество датчиков и высокая частота дискретизации, что порождает очень большой объем данных, требующий для обработки много вычислительных мощностей. Ввиду разреженного профиля получаемого изображения имеет место сильная избыточность собираемой информации. В этой работе предложен прототип эффективной технологии по сбору и реконструкции изображений ультразвуковой томографии, основанной на принципах рандомизированных алгоритмов, которая позволяет сократить время исследования без потерь в качестве. 
\end{abstract}

\section{Актуальность}
Ультразвуковая томография по качеству получаемых снимков достигла сопоставимого с МРТ уровня и активно применяется для исследования мягких тканей \cite{hopp2014breast}.
Преимуществами УЗИ являются относительно низкая стоимость оборудования и обслуживания, безопасность для организма, неинвазивность техники исследования. Для повышения качества и разрешающей способности получаемого в ходе исследования изображения требуется увеличение количества используемых датчиков и частоты дискретизации сигнала. Всё это ведёт к значительному увеличению объема передаваемых данных, что усложняет как производственный процесс, так и проведение диагностики.\\
Опухоли преимущественно имеют более высокую скорость прохождения ультразвука, чем окружающие ткани. Это делает возможным реконструкцию плотностей тканей в исследуемой зоне с помощью уравнений с участием предполагаемых путей распространения сигнала и временем его прибытия на датчики кругового массива (\textit{travel time tomography}) \cite{quan2007sound}. Для метода томографии travel-time типична квадратичная от числа сенсоров зависимость получаемых "сырых" данных для анализа: последовательно с каждого датчика пускается ультразвуковой импульс, который принимают остальные $k-1$ сенсоров. Это приводит к серьезному повышению требований к вычислительной части устройства томографа, а также к увеличению времени обработки.

Примером современного коммерческого ультразвукового томографа является The SoftVue\cite{roy2013breast}, который имеет следующие технические характеристики системы:
\begin{itemize}
\item Мастер сервер: 2 процессора quad-core Intel Xeon E5620, 192 ГБ ОЗУ
\item Сервер реконструкции: 2 процессора quad-core Intel Xeon E5620, 96 ГБ ОЗУ, 2 ГП Nvidia Tesla M2070
\end{itemize}
Так в таблице \ref{table:datasize_ex} производители описали количество данных, получаемоых с датчиков за один срез. На обработку такого среза при использовании только одного сервера для реконструкции требуется несколько минут\cite{roy2013breast}. При этом общее количество таких срезов равно 70. 

\begin{table}[h]
\caption{Объем исходных данных в ГБ за один срез в зависимости от числа датчиков в массиве и частоты дискретизации сигнала \cite{roy2013breast}}
\begin{center}\begin{tabular}{ c | c | c | c }
    \hline
    {Частота дискретизации (МГц)} & \multicolumn{3}{c}{Число датчиков}  \\ \cline{2-4}
    & 256 & 51 & 1024 \\
    
    \hline
    10 & 0.21 & 0.86 & 3.44 \\
    12 & 0.26 & 1.03 & 4.13 \\
    14 & 0.30 & 1.20 & 4.81 \\
    \hline
\end{tabular}\end{center}

\label{table:datasize_ex}
\end{table}

\section{Travel-time томография}
Техника томографии по времени прибытия сигнала уже хорошо изучена и освещена в работах \cite{Kunyansky2012111}, \cite{quan2007sound}, \cite{hopp2014breast}. Основная цель акустической томографии -- восстановить параметры неизвестной среды изучая характеристики распространения звука в ней. Во-первых, для этого требуется точная модель, хорошо описывающая лежащую в её основе физическую систему, и, во-вторых, высокоточные измерения. Тогда решением обратной задачи составляется оценка неизвестной модели. Корректность физической модели, точность измерений и выбор метода решения обратной задачи имеют прямое влияние на качество реконструкции. \\


\section{Compressive Sensing}
Техника обработки сигналов \textit{Compressive Sensing}, или \textit{Опознание со сжатием}, может быть использована для решения проблемы передачи и обработки большого количества информации с массива датчиков, поскольку данная парадигма вводит логарифмическую зависимость использования данных от всего объема. Этот метод уже успешно применяется в магнитно-резонансной(МР) и фотоакустической томографиях \cite{lustig2008compressed}\cite{lustig2007sparse}. Однако, получаемые при УЗ-томографии сигналы содержат информацию в ином представлении по сравнению с МР или фотоакустической томографией. 

\section{Основной результат}
Разработан метод, позволяющий при разработке ультразвукового томографа эффективно внедрить в применяемые алгоритмы по сбору и обработке данных парадигмы Compressive Sensing, способной существенно снизить требования к вычислительным мощностям прибора. 
Для численных экспериментов было проведено моделирование на маломасштабных моделях, которые показали:
\begin{itemize}
\item при использовании томографа с 72 датчиками объем "сырых" данных для последующей реконструкции можно сократить на $\approx 10$\%, уменьшив при этом время, затраченное на получение изображения на $\approx 2$\%
\item
 при увеличении числа датчиков до 100 выигрыш по использованному объему данных и времени составляет $\approx 30$\% и $\approx 15$\%, соответственно
\end{itemize}

\section{Заключение}
Стремительно набирающая популярность парадигма Compressive Sensing показала высокую эффективность при использовании в ультразвуковой томографии. Был разработан метод, позволяющий при конструировании ультразвукового томографа эффективно внедрить в применяемые алгоритмы по сбору и обработке данных парадигмы опознания со сжатием, способной существенно снизить требования к вычислительным мощностям прибора. 


\renewcommand\refname{Литература}
\begin{thebibliography}{8}

\bibitem{hopp2014breast} Breast imaging with 3D ultrasound computer tomography: results of a first in-vivo study in comparison to MRI images / Torsten Hopp, Lukas Šroba, Michael Zapf et al. // Breast Imaging. –– Springer, 2014. –– P. 72–79.

\bibitem{roy2013breast} Breast imaging using ultrasound tomography: From clinical requirements to system design / Olivier Roy, Signe Schmidt, Cuiping Li et al. // Ultrasonics Symposium (IUS), 2013 IEEE International / IEEE. –– 2013. –– P. 1174–1177

\bibitem{quan2007sound} Quan Youli, Huang Lianjie. Sound-speed tomography using first-arrival transmission ultrasound for a ring array // Medical Imaging / International Society for Optics and Photonics. –– 2007. –– P. 651306–651306.

\bibitem{Kunyansky2012111} Kunyansky Leonid. Fast reconstruction algorithms for the thermoacoustic tomography in certain domains with cylindrical or spherical symmetries // Inverse Problems and Imaging. –– 2012. ––Vol. 6, no. 1. –– P. 111–131.

\bibitem{lustig2008compressed} Compressed sensing MRI / Michael Lustig, David L Donoho, Juan M Santos, John M Pauly // Signal Processing Magazine, IEEE. –– 2008. –– Vol. 25, no. 2. –– P. 72–82.

\bibitem{lustig2007sparse}  Lustig Michael, Donoho David, Pauly John M. Sparse MRI: The application of compressed sensing for rapid MR imaging // Magnetic resonance in medicine. –– 2007. –– Vol. 58, no. 6. –– P. 1182–1195.


\end{thebibliography}

\end{document}
