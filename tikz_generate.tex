\documentclass{article}

\usepackage{amsfonts}
\usepackage{amssymb,amsmath,latexsym}
\usepackage{amsthm}
\usepackage[T2A]{fontenc}
\usepackage[utf8]{inputenc}  
\usepackage[english,russian]{babel}

\usepackage[dvips]{graphics}
\usepackage[dvips]{graphicx}
\usepackage[dvips]{color}
\usepackage{curves}
\usepackage{indentfirst}
\usepackage{epsfig}             % MUST!!!
\usepackage{cite}

\usepackage{multirow}
\usepackage{graphicx}
\usepackage{enumitem}
\usepackage{subcaption}
\usepackage{pgfplots}
\usepackage{csvsimple}

\usepackage{tikz} % To generate the plot from csv
\pgfplotsset{compat=newest} % Allows to place the legend below plot
\usepgfplotslibrary{units} % Allows to enter the units nicely
\usepackage{csvsimple}
\usepackage{pgfplots}

\tikzset{font={\fontsize{20pt}{12}\selectfont}}

% set up externalization
\usetikzlibrary{external}
\tikzset{external/system call={latex \tikzexternalcheckshellescape -halt-on-error
-interaction=batchmode -jobname "\image" "\texsource";
dvips -o "\image".ps "\image".dvi;
ps2eps "\image.ps"}}
\tikzexternalize

\begin{document}
    % \begin{tikzpicture}
    %     \begin{axis}[
    %     width=1.0\textwidth,
    %     height=\axisdefaultheight,
    %     view={0}{90},
    %     grid=major, 
    %     % scale only axis,
    %     %xtick={0, 400, 800, 1200, 1600},
    %     xlabel={Порядковый номер датчика},
    %     ylabel={Время, с}]
    %     \addplot+[line width=3pt, mark=none] table [x=x, y=y, col sep=comma, mark=none] {csv/arrivals.csv};
    %     % \addplot+[dashed, mark=none] table [x=x, y=original, col sep=comma, mark=none] {csv/rmse_seq.csv}; 
    %     % \legend{CS-i.i.d., Legacy}
    %     \end{axis}
    % \end{tikzpicture}
    \begin{tikzpicture}
         \begin{axis}[
         width=\textwidth,
         height=\axisdefaultheight,
         view={0}{90},
         grid=major, 
         % scale only axis,
         xtick={0, 500, 1000, 1500, 2000},
         xlabel={Количество датчиков в устройстве},
         ylabel={$m / M \times 100\% $}]
         \addplot+[line width=2pt, mark=none] table [x=x, y=y, col sep=comma, mark=none] {csv/percents_estimation.csv};
         \end{axis}
     \end{tikzpicture}
\end{document}





% \begin{figure}[h]
%     \centering
%     \begin{tikzpicture}
%         \begin{axis}[
%         width=.9\textwidth,
%         height=\axisdefaultheight,
%         view={0}{90},
%         grid=major, 
%         xtick={0, 100, 200, ..., 600},
%         xlabel={Число использованных коэффициентов результата вейвлет-трансформации},
%         ylabel={Ошибка (RMSE), м/с}]
%         \addplot+[mark=none] table [x=coef, y=value, col sep=comma, mark=none] {csv/rmse_coef_in_use_reconstr.csv};
%         \end{axis}
%     \end{tikzpicture}
%     \begin{tikzpicture}
%         \begin{axis}[
%         width=.9\textwidth,
%         height=\axisdefaultheight,
%         view={0}{90},
%         grid=major, 
%         % scale only axis,
%         xlabel={Коэффициенты вейвлет преобразования в отсортированном порядке},
%         ylabel={Амплитуда}]
%         \addplot+[mark=none] table [x=coef, y=value, col sep=comma, mark=none] {csv/line_coefs_sparse_reconstr.csv};
%         \end{axis}
%     \end{tikzpicture}
%     \caption{Ошибка восстановления карты скоростей искусственной модели при использовании неполного набора коэффициентов вейвлет-преобразования (сверху) и набор коэффициентов вейвлет-преобразования (снизу)}
%     \label{fig:used_coeff}  
% \end{figure}

% \begin{figure}[h!]
%     \centering
%     \begin{tikzpicture}
%         \begin{axis}[
%         width=.9\textwidth,
%         height=\axisdefaultheight,
%         view={0}{90},
%         grid=major, 
%         % scale only axis,
%         xtick={0, 400, 800, 1200, 1600},
%         xlabel={Число использованных коэффициентов результата вейвлет-трансформации},
%         ylabel={Ошибка (RMSE), м/с}]
%         \addplot+[mark=none] table [x=coef, y=value, col sep=comma, mark=none] {csv/rmse_coef_in_use_real-img.csv};
%         \end{axis}
%     \end{tikzpicture}
%     \begin{tikzpicture}
%         \begin{axis}[
%         width=.9\textwidth,
%         height=\axisdefaultheight,
%         view={0}{90},
%         grid=major, 
%         % scale only axis,
%         %xtick={0, 400, 800, 1200, 1600},
%         xlabel={Порядковый номер коэффициента},
%         ylabel={Амплитуда}]
%         \addplot+[mark=none] table [x=coef, y=value, col sep=comma, mark=none] {csv/line_coefs_sparse_real-img.csv};
%         \end{axis}
%     \end{tikzpicture}
%     \caption{Ошибка восстановления томографического снимка при использовании неполного набора коэффициентов вейвлет-преобразования (сверху) и набор коэффициентов вейвлет-преобразования (снизу)}
%     \label{fig:used_coef_mri}
% \end{figure}